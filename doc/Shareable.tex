% LaTeX document produced by pod2latex from "Shareable.pm.pod".
% The followings need be defined in the preamble of this document:
%\def\C++{{\rm C\kern-.05em\raise.3ex\hbox{\footnotesize ++}}}
%\def\underscore{\leavevmode\kern.04em\vbox{\hrule width 0.4em height 0.3pt}}
%\setlength{\parindent}{0pt}

\section{SHAREABLE.PM}%
\index{SHAREABLE.PM}

\subsection*{NAME}
IPC::Shareable --- share Perl variables between processes

\subsection*{SYNOPSIS}\begin{verbatim}
  use IPC::Shareable;
  tie($scalar, IPC::Shareable, $glue, { %options });
  tie(%hash, IPC::Shareable, $glue, { %options });
  (tied %hash)->shlock;
  (tied %hash)->shunlock;
\end{verbatim}

\subsection*{CONVENTIONS}%
\index{CONVENTIONS}

The occurrence of a number in square brackets, as in {\tt [}N{\tt ]}, in the text
of this document refers to a numbered note in the the {\tt NOTES} entry elsewhere in this document.

\subsection*{DESCRIPTION}
IPC::Shareable allows you to tie a a variable to shared memory making
it easy to share the contents of that variable with other Perl
processes.  Currently either scalars or hashes can be tied; tying of
arrays remains a work in progress.  However, the variable being tied
may contain arbitrarily complex data structures --- including references
to arrays, hashes of hashes, etc.  See the {\tt REFERENCES} entry elsewhere in this documentbelow for
more information.

The association between variables in distinct processes is provided by
{\em \$glue\/}.  This is an integer number or 4 character string{\tt [}1{\tt ]} that serves
as a common identifier for data across process space.  Hence the
statement
\begin{verbatim}
        tie($scalar, IPC::Shareable, 'data');
\end{verbatim}

in program one and the statement
\begin{verbatim}
        tie($variable, IPC::Shareable, 'data');
\end{verbatim}

in program two will bind \$scalar in program one and \$variable in
program two.  There is no pre-set limit to the number of processes
that can bind to data; nor is there a pre-set limit to the size or
complexity of the underlying data of the tied variables{\tt [}2{\tt ]}.

The bound data structures are all linearized (using Raphael Manfredi's
Storable module) before being slurped into shared memory.  Upon
retrieval, the original format of the data structure is recovered.
Semaphore flags are used for versioning and managing a per-process
cache, allowing quick retrieval of data when, for instance, operating
on a {\em tie()\/}d variable in a tight loop.

\subsection*{OPTIONS}%
\index{OPTIONS}

Options are specified by passing a reference to a hash as the fourth
argument to the tie function that enchants a variable.  Alternatively
you can pass a reference to a hash as the third argument;
IPC::Shareable will then look at the field named {\em 'key'\/} in this hash
for the value of {\em \$glue\/}.  So,
\begin{verbatim}
        tie($variable, IPC::Shareable, 'data', \%options);
\end{verbatim}

is equivalent to
\begin{verbatim}
        tie($variable, IPC::Shareable,
            { 'key' => 'data', ... });
\end{verbatim}

When defining an options hash, values that match the word {\em 'no'\/} in a
case-insensitive manner are treated as false.  Therefore, setting
{\tt \$options\{'create'\} = 'No';} is the same as {\tt \$options\{'create'\} =
0;}.

The following fields are recognized in the options hash.

\begin{description}

\item[key]%
\index{key@key}%

The {\em 'key'\/} field is used to determine the {\em \$glue\/} if {\em \$glue\/} was
not present in the call to {\em tie()\/}.  This argument is then, in turn,
used as the KEY argument in subsequent calls to {\em shmget()\/} and {\em semget()\/}.
If this field is not provided, a value of IPC\underscore{}PRIVATE is assumed,
meaning that your variables cannot be shared with other
processes. (Note that setting {\em \$glue\/} to 0 is the same as using
IPC\underscore{}PRIVATE.)

\item[create]%
\index{create@create}%
\hfil\\
If {\em 'create'\/} is set to a true value, IPC::Shareable will create a new
binding associated with {\em \$glue\/} if such a binding does not already
exist.  If {\em 'create'\/} is false, calls to {\em tie()\/} will fail (returning
undef) if such a binding does not already exist.  This is achieved by
ORing IPC\underscore{}PRIVATE into FLAGS argument of calls to {\em shmget()\/} when
{\em create\/} is true.

\item[exclusive]%
\index{exclusive@exclusive}%
\hfil\\
If {\em 'exclusive'\/} field is set to a true value, calls to {\em tie()\/} will
fail (returning undef) if a data binding associated with {\em \$glue\/}
already exists.  This is achieved by ORing IPC\underscore{} IPC\underscore{}EXCL into the
FLAGS argument of calls to {\em shmget()\/} when {\em 'exclusive'\/} is true.

\item[mode]%
\index{mode@mode}%
\hfil\\
The {\em mode\/} argument is an octal number specifying the access
permissions when a new data binding is being created.  These access
permission are the same as file access permissions in that 0666 is
world readable, 0600 is readable only by the effective UID of the
process creating the shared variable, etc.  If not provided, a default
of 0666 (world readable and writable) will be assumed.

\item[destroy]%
\index{destroy@destroy}%
\hfil\\
If set to a true value, the data binding will be destroyed when the
process calling {\em tie()\/} exits (gracefully){\tt [}3{\tt ]}.

\end{description}

\subsection*{LOCKING}%
\index{LOCKING}

Shareable provides methods to implement application-level locking of
the shared data structures.  These methods are called {\em shlock()\/} and
{\em shunlock()\/}.  To use them you must first get the tied object, either by
saving the return value of the original call to {\em tie()\/} or by using the
built-in {\em tied()\/} function.

To lock a variable, do this:
\begin{verbatim}
  $knot = tie($scalar, IPC::Shareable, $glue, { %options });
  ...
  $knot->shlock;
\end{verbatim}

or equivalently
\begin{verbatim}
  tie($scalar, IPC::Shareable, $glue, { %options });
  (tied $scalar)->shlock;
\end{verbatim}

This will place an exclusive lock on the data of \$scalar.

To unlock a variable do this:
\begin{verbatim}
  $knot->shunlock;
\end{verbatim}

or
\begin{verbatim}
  (tied $scalar)->shunlock;
\end{verbatim}

Note that there is no mechanism for shared locks, but you're probably
safe to rely on Shareable's internal locking mechanism in situations
that would normally call for a shared lock so that's not a big
drawback.  In general, a lock only needs to be applied during a
non-atomic write operation.  For instance, a statement like
\begin{verbatim}
  $scalar = 10;
\end{verbatim}

doesn't really need a lock since it's atomic.  However, if you want to
increment, you really should do
\begin{verbatim}
  (tied $scalar)->shlock;
  ++$scalar;
  (tied $scalar)->shunlock;
\end{verbatim}

since ++\$scalar is non-atomic.

Read-only operations are (I think) atomic so you don't really need to
lock for them.

There are some pitfalls regarding locking and signals that you should
make yourself aware of; these are discussed in the {\tt NOTES} entry elsewhere in this document.

\subsection*{REFERENCES}%
\index{REFERENCES}

If a variable {\em tie()\/}d to Shareable contains references, Shareable acts
in different ways depending upon the initial state of the thingy being
referenced.

\subsubsection*{The Thingy Referenced Is Initially False}%
\index{Thingy Referenced Is Initially False}

If Shareable encounters in a {\em tie()\/}d variable a reference to an empty
hash or a scalar with a false value, Shareable will attempt to {\em tie()\/} the
hash or scalar being referenced.  If a reference is to an empty array,
Shareable defaults to its other behaviour described below since
Shareable cannot {\em tie()\/} arrays.

References to empty hashes can occur whenever a {\em tie()\/}d variable is
cast in a context that forces references to "spring into existence".
Consider, for instance, the following assignment to a {\em tie()\/}d \%hash:
\begin{verbatim}
    $hash{'foo'}{'bar'} = 'xyzzy';
\end{verbatim}

This statement assigns assigns to \$hash\{'foo'\} a reference to an
anonymous hash.  In the anonymous hash it assigns to the key 'bar' the
value 'xyzzy'. Since \%hash is {\em tie()\/}d, the assignment triggers
Shareable, but when Shareable is called, the anonymous hash is still
empty.  Shareable then immediately {\em tie()\/}s the anonymous hash so that
when the assignment \{ 'bar' = 'xyzzy' \} is made, Shareable can catch
it.

One consequence of this behaviour is a statement like
\begin{verbatim}
    $scalar = {};
\end{verbatim}

will, for a {\em tie()\/}d \$scalar, Shareable to {\em tie()\/} the anonymous hash.
Consider this a supported bug.  It does, however mean that statements like
\begin{verbatim}
    $scalar->{'foo'} = 'bar';
\end{verbatim}

should work as expected.

Be warned, however, that each variable {\em tie()\/}d to Shareable requires (at
least) one shared memory segment and one set of three semaphores.  If
you use this feature too liberally, you can find yourself running out
of semaphores quickly.  If that happens to you, consider resorting to
Shareable other behaviour described in the following section.

Another potential problem at the time of writing with using this
behaviour is that locking using {\em shlock()\/} and {\em shunlock()\/} is unreliable.
This is because a data structure spans more than one {\em tie()\/}d variable.  It
is advisable to implement your own locking mechanism if you plan on using
this behaviour of Shareable.

\subsubsection*{The Thingy Referenced Is Initially True}%
\index{Thingy Referenced Is Initially True}

If Shareable encounters in a {\em tie()\/}d variable a reference to a hash with
any key/value pairs, a reference to a true scalar, or a reference to
any array, the contents of the referenced thingy are slurped into the
same shared memory segment as the original {\em tie()\/}d variable.  What that
means is that a statement like
\begin{verbatim}
    $scalar = [ 0 .. 9 ];
\end{verbatim}

makes the contents of the anonymous array referenced by a {\em tie()\/}d \$scalar
visible to other processes.

The good side of this behaviour is that a data structure can be
arbitrarily complex and still only require one set of three
semaphores.  The downside becomes evident when you try to modify the
contents of such a referenced thingy, either in the original process
or elsewhere.  A statement like
\begin{verbatim}
    push(@$scalar, 10, 11, 12);
\end{verbatim}

modifies only the untied anonymous array referenced by \$scalar and not
the {\em tie()\/}d \$scalar itself.  Subsequently, the change to the anonymous
array would be visible only in the process making this statement.

A workaround is to remember which variable is really {\em tie()\/}d and to make
sure you assign into that variable every time you change a thingy that
it references.  An alternative to the above statement that works is
\begin{verbatim}
    $scalar = [ (@$scalar, 10, 11, 12) ];
\end{verbatim}

\subsection*{EXAMPLES}
In a file called {\bf server}:
\begin{verbatim}
    #!/usr/bin/perl -w
    use IPC::Shareable;
    $glue = 'data';
    %options = (
        'create' => 'yes',
        'exclusive' => 'no',
        'mode' => 0644,
        'destroy' => 'yes',
    );
    tie(%colours, IPC::Shareable, $glue, { %options }) or
        die "server: tie failed\n";
    %colours = (
        'red' => [
             'fire truck',
             'leaves in the fall',
        ],
        'blue' => [
             'sky',
             'police cars',
        ],
    );
    (print("server: there are 2 colours\n"), sleep 5)
        while scalar keys %colours == 2;
    print "server: here are all my colours:\n";
    foreach $colour (keys %colours) {
        print "server: these are $colour: ",
            join(', ', @{$colours{$colour}}), "\n";
    }
    exit;
\end{verbatim}

In a file called {\bf client}
\begin{verbatim}
    #!/usr/bin/perl -w
    use IPC::Shareable;
    $glue = 'data';
    %options = (
        'key' => 'paint',
        'create' => 'no',
        'exclusive' => 'no',
        'mode' => 0644,
        'destroy' => 'no',
        );
    tie(%colours, IPC::Shareable, $glue, { %options }) or
        die "client: tie failed\n";
    foreach $colour (keys %colours) {
        print "client: these are $colour: ",
            join(', ', @{$colours{$colour}}), "\n";
    }
    delete $colours{'red'};
    exit;
\end{verbatim}

And here is the output (the sleep commands in the command line prevent
the output from being interrupted by shell prompts):
\begin{verbatim}
    bash$ ( ./server & ) ; sleep 10 ; ./client ; sleep 10
    server: there are 2 colours
    server: there are 2 colours
    server: there are 2 colours
    client: these are blue: sky, police cars
    client: these are red: fire truck, leaves in the fall
    server: here are all my colours:
    server: these are blue: sky, police cars
\end{verbatim}

\subsection*{RETURN VALUES}%
\index{RETURN VALUES}

Calls to {\em tie()\/} that try to implement IPC::Shareable will return true
if successful, {\em undef\/} otherwise.  The value returned is an instance
of the IPC::Shareable class.

\subsection*{INTERNALS}%
\index{INTERNALS}

When a variable is {\em tie()\/}d, a blessed reference to a SCALAR is created.
(This is true even if it is a HASH being {\em tie()\/}d.)  The value thereby
referred is an integer{\tt [}4{\tt ]} ID that is used as a key in a hash called
{\em \%IPC::Shareable::Shm\underscore{}Info\/}; this hash is created and maintained by
IPC::Shareable to manage the variables it has {\em tie()\/}d.  When
IPC::Shareable needs to perform an operation on a {\em tie()\/}d variable, it
dereferences the blessed reference to perform a lookup in
{\em \%IPC::Shareable::Shm\underscore{}Info\/} for the information needed to proceed.

{\em \%IPC::Shareable::Shm\underscore{}Info\/} has the following structure:
\begin{verbatim}
    %IPC::Shareable::Shm_Info = (
\end{verbatim}
\begin{verbatim}
        # - The ID of an enchanted variable
        $id => {
\end{verbatim}
\begin{verbatim}
            # -  A literal indicating the variable type
            'type' => 'SCALAR' || 'HASH',
\end{verbatim}
\begin{verbatim}
            # - The I<$glue> used when tie() was called
            'key' => $glue,
\end{verbatim}
\begin{verbatim}
            # - Shm segment IDs for this variable
            'frag_id' => {
                '0' => $id_1, # - ID of first shm segment
                '1' => $id_2, # - ID of next shm segment
                ... # - etc
            },
\end{verbatim}
\begin{verbatim}
            # - ID of associated semaphores
            'sem_id' => $semid,
\end{verbatim}
\begin{verbatim}
            # - The options passed when tie() was called
            'options' => { %options },
\end{verbatim}
\begin{verbatim}
            # - The value of FLAGS for shmget() calls.
            'flags' => $flags,
\end{verbatim}
\begin{verbatim}
            # - Destroy shm segements on exit?
            'destroy' => $destroy,
                    ;
            # - The version number of the cached data
            'version' => $version,
\end{verbatim}
\begin{verbatim}
            # - A flag that indicates if this process
            # - has a lock on this variable
            'lock' => $lock_flag,
\end{verbatim}
\begin{verbatim}
            # - A flag that indicates whether an
            # - iteration of this variable is in
            # - progress and we should use the local
            # - cache only until the iteration is over.
            # - Meaningless for scalars.
            'hash_iterating' => $iteration_flag,
\end{verbatim}
\begin{verbatim}
            # - Data cache; data will be retrieved from
            # - here when this process's version is the
            # - same as the public version, or when we
            # - have a hash in the middle of some kind
            # - of iteration
            'DATA' => {
                # - User data; where the real
                # - information is stored
                'user' => \$data || \%data,
                # - Internal data used by Shareable to
                # - attach to any thingies referenced
                # - by this variable; see REFERENCES
                # - above
                'internal => {
                    # - Identifier of a thingy attached
                    # - to this variable
                    $string_1 => {
                        # - The shared memory id of the
                        # - attached thingy
                        'shm_id' => $attached_shmid,
                        # - The $glue used when tie()ing
                        # - to this thingy
                        'key' => $glue,
                        # - Type of thingy to attach to
                        'ref_type' => $type,
                        # - Where to store the reference
                        # - to this thingy
                        'hash_key' => $hash_key,
                    },
                    $string_2 => {
                        ... # - Another set of keys like
                            # - $string_1
                    },
                    ... # - Additional $string_n's if
                        # - need be.
                },
            },
\end{verbatim}
\begin{verbatim}
            # - List of associated data structures, and 
            # - flags that indicate if this process has
            # - successfully attached to them
            'attached' => {
                $string_1 => $attached_flag1,
                $string_2 => $attached_flag2,
            },
\end{verbatim}
\begin{verbatim}
            
            },
       ... # - IDs of additional tie()d variables
   );
\end{verbatim}

Perhaps the most important thing to note the existence of the
{\em 'DATA'\/} and {\em 'version'\/} fields: data for all {\em tie()\/}d variables is
stored locally in a per-process cache.  When storing data, the values
of the semaphores referred to by {\em \$Shm\underscore{}Info\{\$id\}\{'sem\underscore{}id'\}\/} are
changed to indicate to the world a new version of the data is
available. When retrieving data for a {\em tie()\/}d variables, the values of
these semaphores are examined to see if another process has created a
more recent version than the cached version.  If a more recent version
is available, it will be retrieved from shared memory and used. If no
more recent version has been created, the cached version is used.

Also stored in the {\em 'DATA'\/} field is a structure that identifies any
"magically created" {\em tie()\/}d variables associated with this variable.
These variables are created by assignments like the following:
\begin{verbatim}
    $hash{'foo'}{'bar'} = 'xyzzy';
\end{verbatim}

See the {\tt REFERENCES} entry elsewhere in this documentfor a complete explanation.

Another important thing to know is that IPC::Shareable allocates
shared memory of a constant size SHM\underscore{}BUFSIZ, where SHM\underscore{}BUFSIZ is
defined in this module.  If the amount of (serialized) data exceeds
this value, it will be fragmented into multiple segments during a
write operation and reassembled during a read operation.

Lastly, if notice that if you {\em tie()\/} a hash and begin
iterating over it, you will get data from and write to
your local cache until Shareable thinks you've reached
the end of the iteration.  At this point Shareable
writes out the entire contents of your hash to shared
memory.  This is done so you can safely iterate via
{\em keys()\/}, {\em values()\/}, and {\em each()\/} without having to worry
about somebody else clobbering a key in the middle of
the loop.

\subsection*{AUTHORS}
Benjamin Sugars $<$bsugars@canoe.ca$>$

Maurice Aubrey $<$maurice@hevanet.com$>$

\subsection*{NOTES}
\subsubsection*{Footnotes from the above sections}%
\index{Footnotes from the above sections}

\begin{enumerate}

\item
If {\em \$glue\/} is longer than 4 characters, only the 4 most significant
characters are used.  These characters are turned into integers by {\em unpack()\/}ing
them.  If {\em \$glue\/} is less than 4 characters, it is space padded.

\item
IPC::Shareable provides no pre-set limits, but the system does.
Namely, there are limits on the number of shared memory segments that
can be allocated and the total amount of memory usable by shared
memory.

\item
If the process has been smoked by an untrapped signal, the binding
will remain in shared memory.  If you're cautious, you might try
\begin{verbatim}
    $SIG{INT} = \&catch_int;
    sub catch_int {
        exit;
    }
    ...
    tie($variable, IPC::Shareable, 'data',
        { 'destroy' => 'Yes!' });
\end{verbatim}

which will at least clean up after your user hits CTRL-C because
IPC::Shareable's DESTROY method will be called.  Or, maybe you'd like
to leave the binding in shared memory, so subsequent process can
recover the data...

\item
The integer happens to be the shared memory ID of the first shared
memory segment used to store the variable's data.

\end{enumerate}

\subsubsection*{General Notes}%
\index{General Notes}

\begin{description}

\item[o]
When using {\em shlock()\/} to lock a variable, be careful to guard against
signals.  Under normal circumstances, Shareable's DESTROY method
unlocks any locked variables when the process exits.  However, if an
untrapped signal is received while a process holds an exclusive lock,
DESTROY will not be called and the lock may be maintained even though
the process has exited.  If this scares you, you might be better off
implementing your own locking methods.

\item[o]
The bulk of Shareable's behaviour when dealing with references relies
on undocumented (and possibly unsupported) features of perl.  Changes
to perl in the future could break Shareable.

\item[o]
As mentioned in the {\tt INTERNALS} entry elsewhere in this documentshared memory segments are acquired
with sizes of SHM\underscore{}BUFSIZ.  SHM\underscore{}BUFSIZ's largest possible value is
nominally SHMMAX, which is highly system-dependent.  Indeed, for some
systems it may be defined at boot time.  If you can't seem to {\em tie()\/}
any variables, it may be that SHM\underscore{}BUFSIZ is set a value that exceeds
SHMMAX on your system.  Try reducing the size of SHM\underscore{}BUFSIZ and
recompiling the module.

\item[o]
The class contains a translation of the constants defined in the
$<$sys/ipc.h$>$, $<$sys/shm.h$>$, and $<$sys/sem.h$>$ header files.  These
constants are used internally by the class and cannot be imported into
a calling environment.  To do that, use IPC::SysV instead.  Indeed, I
would have used IPC::SysV myself, but I haven't been able to get it to
compile on any system I have access to :-(.

\item[o]
Use caution when choosing your values of \$glue.  If IPC::Shareable
needs to acquire more shared memory segments (due to a buffer overrun,
or implicit referencing), those shared memory segments will have a
different \$glue than the \$glue supplied by the application.  In
general, \$glues should be well separated: {\bf aaaa} and {\bf zzzz} are good
choices, since they are unlikely to collide, but {\bf aaaa} and {\bf aaab}
could easily collide.

\item[o]
There is a program called {\em ipcs\/}(1/8) that is available on at least
Solaris and Linux that might be useful for cleaning moribund shared
memory segments or semaphore sets produced by bugs in either
IPC::Shareable or applications using it.

\item[o]
IPC::Shareable version 0.20 or greater does not understand the format
of shared memory segments created by earlier versions of
IPC::Shareable.  If you try to tie to such segments, you will get an
error.  The only work around is to clear the shared memory segments
and start with a fresh set.

\item[o]
Set the variable {\em \$IPC::Shareable::Debug\/} to a true value to produce
$\ast$many$\ast$ verbose debugging messages on the standard error (I don't use
the Perl debugger as much as I should... )

\end{description}

\subsection*{CREDITS}%
\index{CREDITS}

Thanks to all those with comments or bug fixes, especially Stephane
Bortzmeyer $<$bortzmeyer@pasteur.fr$>$, Michael Stevens
$<$michael@malkav.imaginet.co.uk$>$, Richard Neal
$<$richard@imaginet.co.uk$>$, Jason Stevens $<$jstevens@chron.com$>$, Maurice
Aubrey $<$maurice@hevanet.com$>$, and Doug MacEachern
$<$dougm@telebusiness.co.nz$>$.

\subsection*{BUGS}
Certainly; this is alpha software. When you discover an
anomaly, send me an email at bsugars@canoe.ca.

\subsection*{SEE ALSO}
{\em perl\/}(1), {\em perltie\/}(1), {\em Storable\/}(3), {\em shmget\/}(2) and other SysV IPC man
pages.

